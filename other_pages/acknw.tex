%\thispagestyle{empty}
%\vspace{-0.95cm}
\oddsidemargin 0.9 cm 
\evensidemargin -.4 cm
\setlength{\textwidth}{152.4 mm}
\begin{center}
{\Large  \textsl Acknowledgements}
\end{center}
\vskip 0.5 cm
Firstly, I express my sincere gratitude to my advisor Prof. Gagan Mohanty for the continuous support of my Ph.D study and related research, for his patience and motivation. His guidance helped me in all the time of research and writing of this thesis. I am happy to have choosen him as my advisor and mentor for my Ph.D study. I also thank him for his continuous moral support that has helped me to remain cool throughout of my PhD. 
 

Besides my advisor, I thank the rest of my thesis committee: Prof. Tariq Aziz, Prof. Sudeshna Banerjee, and Prof. Shashikant Dugad,  Prof. Indranil Mazumdar and Dr. Rishi Sharma for their insightful comments and encouragement as well as for the hard questions which incented me to widen my research from various perspectives.

 

My sincere thanks also go to Dr. Andrew Whitbeck, Dr. Jim Hirschauer and  Dr. Nhan Viet Tran who provided me an opportunity to join their team  at Fermilab, and  gave access to the laboratory and research facilities. I am also thankful for useful discussions and guidance that they provided which have been very crucial to me. I  thank Bibhuti Paria, my senior, for starting a new collaboration with Fermilab, USA  that gave me an oppurtunity to work in  the group. 


I  thank my friends Jacky, Varghese, Bajarang, Muzamil, Soureek, Nairit, Gouranga, Saurabh, Saranya, Soham, Suman, Pallavi, Arvind, Deepanwita, Sandhya, Adiba, Debashish for useful discussions that has helped my understanding throughout my PhD. 

I also thank all the professors of EHEP, TIFR for stimulating discussions that has helped me a lot during my PhD. 

 

Last but not the least, I thank my family: my parents, my wife and my brother  for supporting me morally  throughout writing this thesis and my life in general. 








%\vspace{2.5cm}
%\hspace{9.5cm} 

\clearpage


